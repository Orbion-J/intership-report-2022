%%%% PREAMBULE MACROS FILE %%%%


\PassOptionsToPackage{svgnames}{color}


% Espace automatique après un guillemet fermant
\fr{
  \renewcommand{\fg}{\fg\xspace}
}{}

%% parenthèses, crochets & accolades qui s'adaptent à la taille de ce qu'il y a à l'intérieur
\newcommand{\ap}[1]{\left(#1\right)}        % auto parenthesis
\newcommand{\ab}[1]{\left[#1\right]}        % auto bracket
\newcommand{\acb}[1]{\left\{#1\right\}}     % auto curly bracket


%% Puces pour les listes
\setitemize[1]{label=$\diamond$}
\setitemize[2]{label=$\circ$}
\setitemize[3]{label=$\star$}

\setenumerate[1]{label=\arabic*.}
\setenumerate[2]{label=\theenumi\arabic*.}
\setenumerate[3]{label=\theenumii\arabic*.}
\setenumerate[4]{label=\theenumiii\arabic*.}
% \setenumerate[2]{label=\alph*.}
% \setenumerate[3]{label=\roman*.}

% style d'item description
\renewcommand{\descriptionlabel}[1]{%
  \hspace\labelsep \upshape\bfseries \textit{#1}%
}


% point median
\newcommand\ptm\textperiodcentered

%soulignement
\newcommand\ul\underline


% panneau attention
\newcommand\danger{
    \makebox[1em]{
        \makebox[0pt][s]{
            % \raisebox{.1em}
            {\footnotesize !}
        }
        \makebox[0pt]{
          \raisebox{-0.1em}
          {\color{red}\Large$\bigtriangleup$}
        }
    }
}


%% écrire une remarque
\newcommand{\rk}[1]{$\rhd$ \textbf{Remar\fr{que}{k}} \emph{#1} }
\newcommand{\qrk}[1]{$\rhd$ \emph{#1} } % quick remark : pas de texte


% encadre le texte passe en parametre sur toute la largeur de la page
\newcommand{\longbox}[1]{
    \begin{center}\fbox{\parbox{\textwidth}{
        #1
    }}\end{center}
}


%% signatures
% manuscrite + nom dessous
\newcommand{\sign}
{\includegraphics{preambule/signature_rj_small.png}\\ Robin \bsc{Jourde}}
% manuscrite + nom aligné à droite
\newcommand{\rsign}{\begin{flushright}\sign\end{flushright}}
% nom
\newcommand{\qsign}{Robin \bsc{Jourde}}
% nom aligné à droite
\newcommand{\rqsign}{\begin{flushright}\qsign\end{flushright}}



% email
\newcommand{\mailto}[1]{\href{mailto:#1}{\texttt{#1}}}

% fancy quotes
% usage :
% \begin{fquote}[author][source][date]
%   ...
% \end{fquote}
\definecolor{quotemark}{gray}{0.7}
\makeatletter
\def\fquote{%
    \@ifnextchar[{\fquote@i}{\fquote@i[]}%]
           }%
\def\fquote@i[#1]{%
    \def\tempa{#1}%
    \@ifnextchar[{\fquote@ii}{\fquote@ii[]}%]
                 }%
\def\fquote@ii[#1]{%
    \def\tempb{#1}%
    \@ifnextchar[{\fquote@iii}{\fquote@iii[]}%]
                      }%
\def\fquote@iii[#1]{%
    \def\tempc{#1}%
    \vspace{1em}%
    \noindent%
    \begin{list}{}{%
         \setlength{\leftmargin}{0.1\textwidth}%
         \setlength{\rightmargin}{0.1\textwidth}%
                  }%
         \item[]%
         \begin{picture}(0,0)%
         \put(-15,-5){\makebox(0,0){\scalebox{3}{\textcolor{quotemark}{``}}}}%
         \end{picture}%
         \begingroup\itshape}%
 %%%%********************************************************************
 \def\endfquote{%
 \endgroup\par%
 \makebox[0pt][l]{%
 \hspace{0.8\textwidth}%
 \begin{picture}(0,0)(0,0)%
 \put(15,15){\makebox(0,0){%
 \scalebox{3}{\color{quotemark}''}}}%
 \end{picture}}%
 \ifx\tempa\empty%
    \ifx\tempb\empty%
       \ifx\tempc\empty%
       \else%
          \hfill\rule{100pt}{0.5pt}\\\mbox{}\hfill\tempc%
       \fi%
    \else%
       \ifx\tempc\empty%
          \hfill\rule{100pt}{0.5pt}\\\mbox{}\hfill\emph{\tempb}%
       \else%
          \hfill\rule{100pt}{0.5pt}\\\mbox{}\hfill\emph{\tempb},\ \tempc%
       \fi%
    \fi%
 \else%
    \ifx\tempb\empty%
       \ifx\tempc\empty%
          \hfill\rule{100pt}{0.5pt}\\\mbox{}\hfill\tempa%
       \else%
          \hfill\rule{100pt}{0.5pt}\\\mbox{}\hfill\tempa,\ \tempc%
       \fi%
    \else%
       \ifx\tempc\empty%
          \hfill\rule{100pt}{0.5pt}\\\mbox{}\hfill\tempa,\ \emph{\tempb}%
       \else%
          \hfill\rule{100pt}{0.5pt}\\\mbox{}\hfill\tempa,\ \emph{\tempb},\ \tempc%
       \fi%
    \fi%
 \fi%
 \par%
 \vspace{0.5em}%
 \end{list}%
 }%
 \makeatother
