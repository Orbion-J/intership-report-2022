%%%% PREAMBULE BEAMER FILE %%%%

% This template has been downloaded from:
% http://www.LaTeXTemplates.com

\mode<presentation> {

  % The Beamer class comes with a number of default slide themes
  % which change the colors and layouts of slides. Below this is a list
  % of all the themes, uncomment each in turn to see what they look like.

  % \usetheme{AnnArbor}
  % \usetheme{Antibes}
  % \usetheme{Bergen}
  % \usetheme{Berkeley}
  % \usetheme{Berlin}
  % \usetheme{Darmstadt}
  % \usetheme{Dresden}
  % \usetheme{Frankfurt}
  % \usetheme{Goettingen}
  % \usetheme{Hannover}
  % \usetheme{Ilmenau}
  % \usetheme{JuanLesPins}
  % \usetheme{Marburg}
  % \usetheme{Montpellier}
  % \usetheme{PaloAlto}
  % \usetheme{Pittsburgh}
  % \usetheme{Rochester}
  % \usetheme{Singapore}
  % \usetheme{Szeged}

  % \usetheme{default}       % rien (cadre carré)
  % \usetheme{Boadilla}      % bas3
  \usetheme{CambridgeUS}   % bas3 + haut2
  % \usetheme{Copenhagen}    % bas2
  % \usetheme{Luebeck}       % bas2 + haut2 (cadre/puces carré)
  % \usetheme{Madrid}        % = Boadilla
  % \usetheme{Malmoe}        % = Luebeck
  % \usetheme{Warsaw}        % bas2 + cardes ombre

  % As well as themes, the Beamer class has a number of color themes
  % for any slide theme. Uncomment each of these in turn to see how it
  % changes the colors of your current slide theme.

  % \usecolortheme{albatross}
  % \usecolortheme{beaver}
  % \usecolortheme{beetle}
  % \usecolortheme{crane}
  % \usecolortheme{dove}
  % \usecolortheme{fly}
  % \usecolortheme{lily}
  % \usecolortheme{orchid}
  % \usecolortheme{rose}
  % \usecolortheme{seagull}
  % \usecolortheme{whale}
  % \usecolortheme{wolverine}

  % \usecolortheme{dolphin}
  \usecolortheme{seahorse}

  % \setbeamertemplate{footline} % To remove the footer line in all slides uncomment this line
  % \setbeamertemplate{footline}[frame number] % To replace the footer line in all slides with a simple slide count uncomment this line

  % \setbeamertemplate{navigation symbols}{} % To remove the navigation symbols from the bottom of all slides uncomment this line

  % \setbeamercovered{transparent} % Fait apparaître les animations en grisé (utile pour la conception, mais peut être commenté lors de la remise du document final)

  % Pour utiliser une police à empattements partout
  % \usefonttheme{serif}

  % Pour rajouter la numérotation des frames dans les pieds de page
  % \newcommand*\oldmacro{}%
  % \let\oldmacro\insertshorttitle%
  % \renewcommand*\insertshorttitle{%
  % \oldmacro\hfill%
  % \insertframenumber\,/\,\inserttotalframenumber}


  % section/subsection
  \setbeamertemplate{section in toc}{%
    \quad ({\inserttocsectionnumber})~{\inserttocsection}%
  }
  \setbeamertemplate{subsection in toc}{%
    \quad\quad $\rhd$~{\inserttocsubsection} \\%
  }

  \setbeamertemplate{navigation symbols}{} % remove nav bar

}

% table des matieres (à paramétrer selon le document)
\setcounter{tocdepth}{2}

\newcommand\tocpage{
  \begin{frame}
    \tableofcontents[firstsection = 2 , currentsubsection]
  \end{frame}
}


%% arbres dans beamer (forest)
% permet d'utiliser onslide=<>{} pour animer les arbres (beamer)
\tikzset{onslide/.code args={<#1>#2}{
    \only<#1>{\pgfkeysalso{#2}}
  }}

\tikzset{
  invisible/.style={text opacity=0, draw opacity = 0},
  visible on/.style={alt=#1{}{invisible}},
  alt/.code args={<#1>#2#3}{%
    \alt<#1>{\pgfkeysalso{#2}}{\pgfkeysalso{#3}} % \pgfkeysalso doesn't change the path
  },
}
\forestset{
  visible on/.style={
    for tree={
      /tikz/visible on={#1},
      edge+={/tikz/visible on={#1}}}}}
