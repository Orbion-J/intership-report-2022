%%% PACKAGES %%%
% \usepackage{amssymb}
\usepackage{amsmath}
% \usepackage{mathrsfs}
\usepackage{amsfonts}

\usepackage{amsthm}         % theoremes et preuves
% \usepackage{bussproofs}     % preuves en déduction nat et autres règles

\DeclareMathAlphabet{\mathbbdigit}{U}{bbold}{m}{n} % mathbb pour les chiffres


%%% LETTRES GRECQUES %%%
\renewcommand\phi{\varphi}
\renewcommand\epsilon{\varepsilon}



%%% MACROS %%%
%% Ensembles gras
\newcommand\N{\mathbb N}
\newcommand\Z{\mathbb Z}
\newcommand\D{\mathbb D}
\newcommand\Q{\mathbb Q}
\newcommand\R{\mathbb R}
% \newcommand\C{\mathbb C}
\newcommand\K{\mathbb K}
\newcommand\F{\mathbb F}
\newcommand\E{\mathbb E}
\renewcommand\P{\mathbb P}


%% chiffres gras
\newcommand\zero{\mathbbdigit 0}
\newcommand\one{\mathbbdigit 1}
\newcommand\two{\mathbbdigit 2}



%% Autres ensembles
\newcommand\0{\varnothing}

\newcommand\M{\mathscr M}
\newcommand\MnR{\M_n(\R)}
\newcommand\MnC{\M_n(\C)}
\newcommand\MnK{\M_n(\K)}

\newcommand\B{\mathcal B}
\newcommand\Cl{\mathscr C}
\newcommand\Fn{\mathscr F}

\newcommand\Sfrak{\mathfrak S}



%% Opérateurs classiques
\DeclareMathOperator\id{id}
\DeclareMathOperator\Vect{Vect}
\DeclareMathOperator\card{card}
\DeclareMathOperator\rg{rg}
\DeclareMathOperator\im{im}
\DeclareMathOperator\colim{colim}

%% Catégories
\newcommand\Set{\mathbf{Set}}
\newcommand\Yo{\!\text{\begin{CJK}{UTF8}{min}よ\end{CJK}}}
\newcommand\op{{\,^\text{op}}}
\newcommand{\cattxt}[2]{#1\mathbf{\textbf{-}#2}}
\newcommand{\nlab}[1]{\href{https://ncatlab.org/nlab/show/#1}{[\fr{voir sur le}{see on} nlab]}}

\DeclareMathOperator\st{st}
\DeclareMathOperator\ob{ob}
\DeclareMathOperator\Ran{Ran}
\DeclareMathOperator\Lan{Lan}
\DeclareMathOperator\Kl{Kl}
\DeclareMathOperator\Hom{Hom}

%% Divers

% equivalent
\newcommand{\eq}[1]{\underset{#1}{\sim}}

% fleche de limite
\newcommand{\tend}[2]{\underset{#1 \to #2}{\longrightarrow}} 

% definition d'une fonction
\newcommand{\func}[4]{
    \ \begin{array}{|lll}
        #1 &\longrightarrow& #2 \\
        #3 &\longmapsto& #4
    \end{array}
}

%bijection
\newcommand\bij{\xrightarrow{\sim}}

%autres fleches
\newcommand{\xto}[1]{\xrightarrow{#1}}
\newcommand\too{\multimap}

\newcommand\from{\leftarrow}
% \newcommand\mapsfrom{\mathrel{\reflectbox{\ensuremath{\mapsto}}}}

%d rond
\newcommand\drond{\partial}

% carre de fin de preuve
\renewcommand\qedsymbol{$\square$}
\newcommand\cqfd{\begin{flushright}\qedsymbol\end{flushright}}

%% Theoremes FR
\theoremstyle{plain}
\newtheorem{thm}{\fr{Théorème}{Theorem}}[section]
\newtheorem{lem}[thm]{Lemm\fr{e}{a}}
\newtheorem{cor}[thm]{Corolla\fr{ire}{ry}}
\newtheorem{prop}[thm]{Proposition}
\newtheorem{obj}{\fr{Objectif}{Goal}}


\theoremstyle{definition}
\newtheorem{ex}[thm]{Ex\fr{e}{a} mple}
\newtheorem{dfn}[thm]{D\fr{é}{e} finition}
\newtheorem{nota}[thm]{Notation}

\theoremstyle{remark}
\newtheorem{rem}[thm]{Remar\fr{que}{k}}

%% Disjonctions de cas
\newenvironment{disjonction}{\begin{enumerate}[label=\textbf{Cas\fr{e}{}  \arabic*.}]}{\end{enumerate}}
